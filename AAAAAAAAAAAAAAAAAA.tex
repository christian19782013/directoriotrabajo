%% LyX 2.1.3 created this file.  For more info, see http://www.lyx.org/.
%% Do not edit unless you really know what you are doing.
\documentclass[spanish]{beamer}
\usepackage{mathptmx}
\usepackage[T1]{fontenc}
\usepackage[utf8]{luainputenc}
\usepackage{amsmath}
\usepackage{amssymb}

\makeatletter
%%%%%%%%%%%%%%%%%%%%%%%%%%%%%% Textclass specific LaTeX commands.
 % this default might be overridden by plain title style
 \newcommand\makebeamertitle{\frame{\maketitle}}%
 % (ERT) argument for the TOC
 \AtBeginDocument{%
   \let\origtableofcontents=\tableofcontents
   \def\tableofcontents{\@ifnextchar[{\origtableofcontents}{\gobbletableofcontents}}
   \def\gobbletableofcontents#1{\origtableofcontents}
 }

%%%%%%%%%%%%%%%%%%%%%%%%%%%%%% User specified LaTeX commands.
\usetheme{Warsaw}
% or ...

\setbeamercovered{transparent}
% or whatever (possibly just delete it)

\makeatother

\usepackage{babel}
\addto\shorthandsspanish{\spanishdeactivate{~<>}}

\begin{document}





\title[Título breve alternativo ]{Título tal como aparecerá en la presentación}


\subtitle{Entorno sólo para poner un subtítulo al artículo}


\author[Autor, Otro]{F.~Autor\inst{1} \and S.~Otro\inst{2}}


\institute[Universidades de Algún Sitio y de Otro Sitio]{\inst{1}Departamento de Informática\\
Universidad de Algún Sitio\and \inst{2}Departamento de Filosofía
Teórica\\
Universidad de Otro Sitio}


\date[CFP 2003

]{Conferencia sobre Presentaciones Fabulosas, 2003}

\makebeamertitle


%\pgfdeclareimage[height=0.5cm]{institution-logo}{institución-logo-nombrearchivo}
%\logo{\pgfuseimage{institution-logo}}



\AtBeginSubsection[]{%
  \frame<beamer>{ 
    \frametitle{Índice}   
    \tableofcontents[currentsection,currentsubsection] 
  }
}



%\beamerdefaultoverlayspecification{<+->}
\begin{frame}{Índice}


\tableofcontents{}




\end{frame}

\section{Motivación}


\subsection[Problema básico]{El problema básico que estudiamos}
\begin{frame}{Poner títulos informativos}


\framesubtitle{Los subtítulos de los fotogramas son opcionales.}
\begin{itemize}
\item Usar Enumeración{*} a discreción.


\pause{}

\item Usar oraciones muy cortas o frases cortas.


\pause{}

\item Estos solapados se crean usando el estilo Pausa.
\end{itemize}
\end{frame}

\begin{frame}{Poner títulos informativos }

\begin{itemize}
\item<1-> También se pueden usar estas especificaciones para crear solapados 
\item<3-> Esto permite presentar cosas en cualquier orden
\item<2-> Ésta se muestra en segundo lugar
\end{itemize}
\end{frame}

\begin{frame}{Poner títulos informativos}

\begin{block}<1->{}

\begin{itemize}
\item Bloque sin título.
\item Mostrado en todas las diapositivas.
\end{itemize}
\end{block}
\begin{exampleblock}<2->{Título de algún Bloque de ejemplo}

\begin{itemize}
\item $e^{i\pi}=-1$.
\item $e^{i\pi/2}=i$.
\end{itemize}
\end{exampleblock}
\end{frame}

\subsection{Trabajo previo}
\begin{frame}{Poner títulos informativos}



\begin{example}<1->
En la primera diapositiva.
\end{example}



\begin{example}<2->
En la segunda diapositiva.
\end{example}

\end{frame}

\section{Nuestros resultados/Contribución}


\subsection{Resultados principales}
\begin{frame}{Poner títulos informativos}

\begin{theorem}
En la primera diapositiva.
\end{theorem}


\pause{}
\begin{corollary}
En la segunda diapositiva.
\end{corollary}

\end{frame}

\begin{frame}{Poner títulos informativos }

\begin{columns}[t]


\column{5cm}
\begin{theorem}<1->
En la columna de la izquierda.
\end{theorem}


\column{5cm}
\begin{corollary}<2->
En la columna de la derecha.\\
Nueva línea
\end{corollary}

\end{columns}

\end{frame}

\subsection{Ideas básicas para demostraciones/implementaciones}


\section*{Sumario}
\begin{frame}{Sumario}

\begin{itemize}
\item El \alert{primer mensaje principal }de la exposición en una o dos
líneas.
\item El \alert{segundo mensaje principal} de la exposición en una o dos
líneas.
\item Quizás un \alert{tercer mensaje}, pero no más.
\end{itemize}




\medskip{}

\begin{itemize}
\item Perspectiva

\begin{itemize}
\item Lo que no hemos hecho todavía.
\item Otras cosas pendientes.
\end{itemize}
\end{itemize}
\end{frame}
\appendix

\section*{Apéndice}


\subsection*{Lecturas complementarias}
\begin{frame}[allowframebreaks]{Lecturas complementarias}


\beamertemplatebookbibitems
\begin{thebibliography}{1}
\bibitem{Autor1990}A. Autor. \newblock\emph{Manual de Lo que sea}.\newblock
Editorial, 1990.\beamertemplatearticlebibitems

\bibitem{Alguien2002}S. Alguien.\newblock Sobre esto y aquello\emph{.}
\newblock\emph{Revista Esto y Aquello}. 2(1):50--100, 2000.\end{thebibliography}
\end{frame}

\end{document}
